\chapter*{Wstęp}
\addcontentsline{toc}{chapter}{Wstęp}
\markboth{Wstęp}{Wstęp}

\noindent Współczesna edukacja informatyczna w szkołach podstawowych powinna 
rozwijać praktyczne umiejętności uczniów i przygotowywać ich do świadomego 
korzystania z technologii. Celem niniejszej pracy magisterskiej jest stworzenie kursu programowania 
dla klas 7–8, zintegrowanego z platformą Szkopuł, obejmującego 24 jednostek lekcyjnych, 
zestawy zadań i quizów, a także materiały ewaluacyjne -- w tym klasówki i projekty zespołowe.
Kurs jest zgodny z podstawą programową i może być wykorzystywany w ramach nauczania w szkole.
\\
\noindent Oprócz części dydaktycznej praca obejmuje również rozwój funkcjonalności platformy -- wdrożenie
modułu Omówienia, który umożliwi dodawanie i kontrolę dostępu do komentarzy do zadań. 
Całość została zaprojektowana z myślą o praktycznym podejściu do nauczania informatyki i 
może zostać rozszerzona o elementy sztucznej inteligencji, np. w projektach semestralnych.

Rezultatem pracy będzie:

\begin{itemize}
\item Działający kurs opublikowany na Szkopuł Kursy, 
\item Komplet konspektów i materiałów do każdej opisanej lekcji,
\item Nowy komponent ,,Omówienia'' platformy Szkopuł.
\end{itemize}

\clearpage{\pagestyle{empty}\cleardoublepage}
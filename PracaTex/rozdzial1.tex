\chapter{Z czego składa się kurs? Filary programu.}

Program rozpoczyna się od nauki podstaw programowania, tak aby uczniowie mogli 
rozwijać umiejętność myślenia komputacyjnego na dalszych etapach kursu. 
Każda jednostka dydaktyczna składa się z części teoretycznej (w tym elementów „unplugged”) 
oraz zadań praktycznych realizowanych przy użyciu języka Python.

Ćwiczenia informatyki ,,unplugged'' są według nowoczesnych badań co najmniej tak efektywne,
jak standardowe metody nauki \cite{1}.

Plan zajęć obejmuje 28 godzin lekcyjnych; podana liczba godzin ma charakter orientacyjny i może być dostosowywana w zależności od potrzeb grupy.

\section*{Myślenie algorytmiczne, podstawy programowania (12 godzin)}

Moduł ten koncentruje się na wprowadzeniu uczniów w abstrakcyjne pojęcia programistyczne, takie jak zmienne, 
instrukcje warunkowe, pętle oraz funkcje. Zajęcia te przeplatane są mini-projektami, 
które umożliwiają uczniom praktyczne zastosowanie zdobytej wiedzy oraz wyrażenie własnej kreatywności.

Takie podejście zwiększa szanse na zaangażowanie uczniów, którzy niekoniecznie wykazują naturalne predyspozycje matematyczne. 
Abstrakcyjne koncepty nadal odgrywają istotną rolę, ponieważ rozwijają zdolność adaptacji do dynamicznie zmieniających się technologii 
informatycznych \cite{2}.

Badania wskazują również, że dziewczęta częściej interesują się praktycznym zastosowaniem technologii niż jej konstrukcją techniczną 
\cite{3}. Włączenie projektowych i aplikacyjnych elementów do zajęć może więc pozytywnie wpłynąć na ich aktywność i
zaangażowanie w naukę informatyki.

\begin{enumerate}
    \item Podstawy Pythona – zmienne, operatory, \texttt{print()}, \texttt{input()}. Po co programować?
    \item Myślenie komputacyjne od kuchni – specyfikacja, algorytm. Co to znaczy rozwiązać problem?
    \item Instrukcje warunkowe – \texttt{if}, \texttt{elif}, \texttt{else}. Jak kierować zachowaniem komputera?
    \item Pętle \texttt{for} i \texttt{while} – iteracje, przerwania (\texttt{break}, \texttt{continue}). Jak sobie poradzić z powtarzalną pracą? \textbf{[podwójna godzina]}
    \item Zastosowanie praktyczne: Quiz. Jak użyć tego co umiem? \textbf{[podwójna godzina]}
    \item Funkcje w Pythonie – argumenty, wartości zwracane, funkcje wbudowane.
    \item Matematyka z pomocą Pythona – Jak napisać program rozwiązujący moją pracę domową?
    \item Lista, słownik – Kiedy zmienna nie wystarcza.
    \item Zastosowanie praktyczne – Sterowanie „robotem”. Biblioteka \texttt{turtle}. \textbf{[podwójna godzina]}
\end{enumerate}

\section*{Kryptografia (4 godziny)}

Choć programowanie stanowi kluczowy element wprowadzający do informatyki, nie wyczerpuje całego spektrum tej dziedziny. 
Włączenie krótkich modułów tematycznych, 
takich jak kryptografia klasyczna, pozwala uczniom lepiej zrozumieć szerokość informatyki jako nauki.
Zgodnie z podejściem stosowanym w szkołach średnich w Szwajcarii \cite{4}, 
kryptografia może być skutecznym sposobem rozwijania umiejętności analitycznego 
myślenia oraz krytycznego podejścia do informacji – nawet bez wcześniejszej wiedzy z zakresu informatyki. 
Szczególną rolę odgrywa tu motywacja uczniów: element rywalizacji między twórcami a łamaczami kodów stanowi silny czynnik angażujący. 
W badaniach nad tego typu modułami wykazano, że aż 70\% uczniów wykazało chęć dalszego zgłębiania tematu,
co czyni z kryptografii znakomity punkt wyjścia do dalszych zajęć.

\begin{enumerate}
    \item Podstawy kryptografii – Poznajemy szyfr Cezara, \texttt{rot13}. Dlaczego są słabym zabezpieczeniem?
    \item Kody ASCII – zmiana liter w liczbę.
    \item Implementacja wybranej metody szyfrowania – podstawieniowej, przesunięć, \textit{leet speak}, Rozier. \textbf{[podwójna godzina]}
\end{enumerate}

\section*{Algorytmika z Pythonem (8 godzin)}
W tej części kursu uczniowie zapoznają się z podstawowymi algorytmami zawartymi w polskiej podstawie programowej dla klas 7–8. 
Analiza algorytmów odbywa się zarówno na poziomie koncepcyjnym (na kartce), jak i implementacyjnym (w języku Python).

Ponieważ algorytmika jest dziedziną o wysokim stopniu abstrakcji, szczególną uwagę poświęca się wizualizacji działania omawianych algorytmów. 
W procesie nauczania wykorzystywane są elementy gier, 
zabaw oraz komputerowe narzędzia wizualizacyjne, które są kluczowe by uczniowie mogli dobrze
zrozumieć mechanizmy stojące za poszczególnymi rozwiązaniami \cite{5}.

\begin{enumerate}
    \item Własności liczbowe – podzielność, suma cyfr, iloczyn cyfr. \textbf{[podwójna godzina]}
    \item Kod binarny, szesnastkowy.
    \item Algorytm Euklidesa – zastosowanie.
    \item Wyszukiwanie w zbiorze – Liniowo znajdujemy minimum, maksimum, k-ty element. \textbf{[podwójna godzina]}
    \item Porządkowanie zbioru – sortowanie przez wybieranie. \textbf{[podwójna godzina]}
\end{enumerate}

\section*{Wstęp do AI (4 godziny)}
Ostatnia część kursu to odpowiedź na rosnącą potrzebę wdrożenia uczniów w tematykę
sztucznej inteligencji. W związku z dynamicznym rozwojem tej dziedziny, zajęcia skupiają
się na zbudowaniu prostego modelu i uczuleniu uczniów na typowy problem uczenia maszynowego
(Garbage In - Garbage Out).

\begin{enumerate}
    \item Czym jest sztuczna inteligencja? Implementujemy uproszczone "drzewo decyzyjne".
    \item Moduł scikit-learn i sprytniejsze drzewa decyzyjne.
    \item Trenowanie własnego modelu i eksploracja danych. \textbf{[podwójna godzina]}
\end{enumerate}

% zostawi� na ko�cu, w razie potrzeby usuwa nag��wki z ostatniej pustej strony 
\clearpage{\pagestyle{empty}\cleardoublepage}